\documentclass{article}

\usepackage{fullpage}
\usepackage{fancyhdr}
\usepackage{amsmath}
\usepackage{amsfonts}
\usepackage{graphicx}
\usepackage{fancyvrb}
\usepackage{color}
\usepackage[ascii]{inputenc}

\makeatletter
\def\PY@reset{\let\PY@it=\relax \let\PY@bf=\relax%
  \let\PY@ul=\relax \let\PY@tc=\relax%
  \let\PY@bc=\relax \let\PY@ff=\relax}
\def\PY@tok#1{\csname PY@tok@#1\endcsname}
\def\PY@toks#1+{\ifx\relax#1\empty\else%
  \PY@tok{#1}\expandafter\PY@toks\fi}
\def\PY@do#1{\PY@bc{\PY@tc{\PY@ul{%
        \PY@it{\PY@bf{\PY@ff{#1}}}}}}}
\def\PY#1#2{\PY@reset\PY@toks#1+\relax+\PY@do{#2}}

\expandafter\def\csname PY@tok@gd\endcsname{\def\PY@tc##1{\textcolor[rgb]{0.63,0.00,0.00}{##1}}}
\expandafter\def\csname PY@tok@gu\endcsname{\let\PY@bf=\textbf\def\PY@tc##1{\textcolor[rgb]{0.50,0.00,0.50}{##1}}}
\expandafter\def\csname PY@tok@gt\endcsname{\def\PY@tc##1{\textcolor[rgb]{0.00,0.27,0.87}{##1}}}
\expandafter\def\csname PY@tok@gs\endcsname{\let\PY@bf=\textbf}
\expandafter\def\csname PY@tok@gr\endcsname{\def\PY@tc##1{\textcolor[rgb]{1.00,0.00,0.00}{##1}}}
\expandafter\def\csname PY@tok@cm\endcsname{\let\PY@it=\textit\def\PY@tc##1{\textcolor[rgb]{0.25,0.50,0.50}{##1}}}
\expandafter\def\csname PY@tok@vg\endcsname{\def\PY@tc##1{\textcolor[rgb]{0.10,0.09,0.49}{##1}}}
\expandafter\def\csname PY@tok@m\endcsname{\def\PY@tc##1{\textcolor[rgb]{0.40,0.40,0.40}{##1}}}
\expandafter\def\csname PY@tok@mh\endcsname{\def\PY@tc##1{\textcolor[rgb]{0.40,0.40,0.40}{##1}}}
\expandafter\def\csname PY@tok@go\endcsname{\def\PY@tc##1{\textcolor[rgb]{0.53,0.53,0.53}{##1}}}
\expandafter\def\csname PY@tok@ge\endcsname{\let\PY@it=\textit}
\expandafter\def\csname PY@tok@vc\endcsname{\def\PY@tc##1{\textcolor[rgb]{0.10,0.09,0.49}{##1}}}
\expandafter\def\csname PY@tok@il\endcsname{\def\PY@tc##1{\textcolor[rgb]{0.40,0.40,0.40}{##1}}}
\expandafter\def\csname PY@tok@cs\endcsname{\let\PY@it=\textit\def\PY@tc##1{\textcolor[rgb]{0.25,0.50,0.50}{##1}}}
\expandafter\def\csname PY@tok@cp\endcsname{\def\PY@tc##1{\textcolor[rgb]{0.74,0.48,0.00}{##1}}}
\expandafter\def\csname PY@tok@gi\endcsname{\def\PY@tc##1{\textcolor[rgb]{0.00,0.63,0.00}{##1}}}
\expandafter\def\csname PY@tok@gh\endcsname{\let\PY@bf=\textbf\def\PY@tc##1{\textcolor[rgb]{0.00,0.00,0.50}{##1}}}
\expandafter\def\csname PY@tok@ni\endcsname{\let\PY@bf=\textbf\def\PY@tc##1{\textcolor[rgb]{0.60,0.60,0.60}{##1}}}
\expandafter\def\csname PY@tok@nl\endcsname{\def\PY@tc##1{\textcolor[rgb]{0.63,0.63,0.00}{##1}}}
\expandafter\def\csname PY@tok@nn\endcsname{\let\PY@bf=\textbf\def\PY@tc##1{\textcolor[rgb]{0.00,0.00,1.00}{##1}}}
\expandafter\def\csname PY@tok@no\endcsname{\def\PY@tc##1{\textcolor[rgb]{0.53,0.00,0.00}{##1}}}
\expandafter\def\csname PY@tok@na\endcsname{\def\PY@tc##1{\textcolor[rgb]{0.49,0.56,0.16}{##1}}}
\expandafter\def\csname PY@tok@nb\endcsname{\def\PY@tc##1{\textcolor[rgb]{0.00,0.50,0.00}{##1}}}
\expandafter\def\csname PY@tok@nc\endcsname{\let\PY@bf=\textbf\def\PY@tc##1{\textcolor[rgb]{0.00,0.00,1.00}{##1}}}
\expandafter\def\csname PY@tok@nd\endcsname{\def\PY@tc##1{\textcolor[rgb]{0.67,0.13,1.00}{##1}}}
\expandafter\def\csname PY@tok@ne\endcsname{\let\PY@bf=\textbf\def\PY@tc##1{\textcolor[rgb]{0.82,0.25,0.23}{##1}}}
\expandafter\def\csname PY@tok@nf\endcsname{\def\PY@tc##1{\textcolor[rgb]{0.00,0.00,1.00}{##1}}}
\expandafter\def\csname PY@tok@si\endcsname{\let\PY@bf=\textbf\def\PY@tc##1{\textcolor[rgb]{0.73,0.40,0.53}{##1}}}
\expandafter\def\csname PY@tok@s2\endcsname{\def\PY@tc##1{\textcolor[rgb]{0.73,0.13,0.13}{##1}}}
\expandafter\def\csname PY@tok@vi\endcsname{\def\PY@tc##1{\textcolor[rgb]{0.10,0.09,0.49}{##1}}}
\expandafter\def\csname PY@tok@nt\endcsname{\let\PY@bf=\textbf\def\PY@tc##1{\textcolor[rgb]{0.00,0.50,0.00}{##1}}}
\expandafter\def\csname PY@tok@nv\endcsname{\def\PY@tc##1{\textcolor[rgb]{0.10,0.09,0.49}{##1}}}
\expandafter\def\csname PY@tok@s1\endcsname{\def\PY@tc##1{\textcolor[rgb]{0.73,0.13,0.13}{##1}}}
\expandafter\def\csname PY@tok@sh\endcsname{\def\PY@tc##1{\textcolor[rgb]{0.73,0.13,0.13}{##1}}}
\expandafter\def\csname PY@tok@sc\endcsname{\def\PY@tc##1{\textcolor[rgb]{0.73,0.13,0.13}{##1}}}
\expandafter\def\csname PY@tok@sx\endcsname{\def\PY@tc##1{\textcolor[rgb]{0.00,0.50,0.00}{##1}}}
\expandafter\def\csname PY@tok@bp\endcsname{\def\PY@tc##1{\textcolor[rgb]{0.00,0.50,0.00}{##1}}}
\expandafter\def\csname PY@tok@c1\endcsname{\let\PY@it=\textit\def\PY@tc##1{\textcolor[rgb]{0.25,0.50,0.50}{##1}}}
\expandafter\def\csname PY@tok@kc\endcsname{\let\PY@bf=\textbf\def\PY@tc##1{\textcolor[rgb]{0.00,0.50,0.00}{##1}}}
\expandafter\def\csname PY@tok@c\endcsname{\let\PY@it=\textit\def\PY@tc##1{\textcolor[rgb]{0.25,0.50,0.50}{##1}}}
\expandafter\def\csname PY@tok@mf\endcsname{\def\PY@tc##1{\textcolor[rgb]{0.40,0.40,0.40}{##1}}}
\expandafter\def\csname PY@tok@err\endcsname{\def\PY@bc##1{\setlength{\fboxsep}{0pt}\fcolorbox[rgb]{1.00,0.00,0.00}{1,1,1}{\strut ##1}}}
\expandafter\def\csname PY@tok@kd\endcsname{\let\PY@bf=\textbf\def\PY@tc##1{\textcolor[rgb]{0.00,0.50,0.00}{##1}}}
\expandafter\def\csname PY@tok@ss\endcsname{\def\PY@tc##1{\textcolor[rgb]{0.10,0.09,0.49}{##1}}}
\expandafter\def\csname PY@tok@sr\endcsname{\def\PY@tc##1{\textcolor[rgb]{0.73,0.40,0.53}{##1}}}
\expandafter\def\csname PY@tok@mo\endcsname{\def\PY@tc##1{\textcolor[rgb]{0.40,0.40,0.40}{##1}}}
\expandafter\def\csname PY@tok@kn\endcsname{\let\PY@bf=\textbf\def\PY@tc##1{\textcolor[rgb]{0.00,0.50,0.00}{##1}}}
\expandafter\def\csname PY@tok@mi\endcsname{\def\PY@tc##1{\textcolor[rgb]{0.40,0.40,0.40}{##1}}}
\expandafter\def\csname PY@tok@gp\endcsname{\let\PY@bf=\textbf\def\PY@tc##1{\textcolor[rgb]{0.00,0.00,0.50}{##1}}}
\expandafter\def\csname PY@tok@o\endcsname{\def\PY@tc##1{\textcolor[rgb]{0.40,0.40,0.40}{##1}}}
\expandafter\def\csname PY@tok@kr\endcsname{\let\PY@bf=\textbf\def\PY@tc##1{\textcolor[rgb]{0.00,0.50,0.00}{##1}}}
\expandafter\def\csname PY@tok@s\endcsname{\def\PY@tc##1{\textcolor[rgb]{0.73,0.13,0.13}{##1}}}
\expandafter\def\csname PY@tok@kp\endcsname{\def\PY@tc##1{\textcolor[rgb]{0.00,0.50,0.00}{##1}}}
\expandafter\def\csname PY@tok@w\endcsname{\def\PY@tc##1{\textcolor[rgb]{0.73,0.73,0.73}{##1}}}
\expandafter\def\csname PY@tok@kt\endcsname{\def\PY@tc##1{\textcolor[rgb]{0.69,0.00,0.25}{##1}}}
\expandafter\def\csname PY@tok@ow\endcsname{\let\PY@bf=\textbf\def\PY@tc##1{\textcolor[rgb]{0.67,0.13,1.00}{##1}}}
\expandafter\def\csname PY@tok@sb\endcsname{\def\PY@tc##1{\textcolor[rgb]{0.73,0.13,0.13}{##1}}}
\expandafter\def\csname PY@tok@k\endcsname{\let\PY@bf=\textbf\def\PY@tc##1{\textcolor[rgb]{0.00,0.50,0.00}{##1}}}
\expandafter\def\csname PY@tok@se\endcsname{\let\PY@bf=\textbf\def\PY@tc##1{\textcolor[rgb]{0.73,0.40,0.13}{##1}}}
\expandafter\def\csname PY@tok@sd\endcsname{\let\PY@it=\textit\def\PY@tc##1{\textcolor[rgb]{0.73,0.13,0.13}{##1}}}

\def\PYZbs{\char`\\}
\def\PYZus{\char`\_}
\def\PYZob{\char`\{}
\def\PYZcb{\char`\}}
\def\PYZca{\char`\^}
\def\PYZam{\char`\&}
\def\PYZlt{\char`\<}
\def\PYZgt{\char`\>}
\def\PYZsh{\char`\#}
\def\PYZpc{\char`\%}
\def\PYZdl{\char`\$}
\def\PYZhy{\char`\-}
\def\PYZsq{\char`\'}
\def\PYZdq{\char`\"}
\def\PYZti{\char`\~}
% for compatibility with earlier versions
\def\PYZat{@}
\def\PYZlb{[}
\def\PYZrb{]}
\makeatother

\setlength\headheight{15pt}
\renewcommand{\headrulewidth}{0pt}


\begin{document}

\thispagestyle{fancy}

\noindent \textbf{Basics of Computer Programs:}
\begin{enumerate}
\item The entry point.
  \begin{enumerate}
  \item The place where execution begins.
  \item Most frequently called the `main' function.
  \item Responsible for the high level organization of the program.
  \end{enumerate}
\item Variables
  \begin{enumerate}
  \item Something that ``holds'' a value.
  \item Like `x' (or any variable) in algebra
  \end{enumerate}
\item Functions
  \begin{enumerate}
  \item A procedure that accepts 0 or more aguments and returns a new value.
  \item We define functions like so:
    \begin{Verbatim}[commandchars=\\\{\}]
      \PY{n}{type} \PY{n+nf}{function}\PY{p}{(}\PY{n}{type} \PY{n}{param1}\PY{p}{,}\PY{n}{type} \PY{n}{param2}\PY{p}{)}\PY{p}{\PYZob{}}
      \PY{c+c1}{// Function body}
      \PY{k}{return} \PY{n}{value}\PY{p}{;}
      \PY{p}{\PYZcb{}}
    \end{Verbatim}
  \item We can call functions like so:
    \begin{Verbatim}[commandchars=\\\{\}]
      \PY{n}{variable} \PY{o}{=} \PY{n}{function}\PY{p}{(}\PY{n}{arg1}\PY{p}{,}\PY{n}{arg2}\PY{p}{)}\PY{p}{;}
    \end{Verbatim}
  \end{enumerate}
\item Logic (Flow of Control) \\
  The flow of control is the logic that the program must preform to choose the
  actions it preforms. \\
  Programming logic uses boolean values. Boolean value are like switches that
  are either ON or OFF.
  \begin{enumerate}
  \item Boolean operators:
    \begin{Verbatim}[commandchars=\\\{\}]
      \PY{n}{a} \PY{o}{=}\PY{o}{=} \PY{n}{b}\PY{p}{;} \PY{c+c1}{// This tests if a is equal to b.}
      \PY{n}{a} \PY{o}{!}\PY{o}{=} \PY{n}{b}\PY{p}{;} \PY{c+c1}{// This tests if a is not  equal to b. This is the same as !(a == b)}
      \PY{o}{!}\PY{n}{a}\PY{p}{;}     \PY{c+c1}{// This test for NOT a.}
      \PY{n}{a} \PY{o}{\PYZam{}}\PY{o}{\PYZam{}} \PY{n}{b}\PY{p}{;} \PY{c+c1}{// This tests if a AND b are true.}
      \PY{n}{a} \PY{o}{|}\PY{o}{|} \PY{n}{b}\PY{p}{;} \PY{c+c1}{// This test if a OR b are true.}
    \end{Verbatim}
  \item The `if' statement: usage:
    \begin{Verbatim}[commandchars=\\\{\}]
      \PY{k}{if} \PY{p}{(}\PY{n}{boolean}\PY{p}{)}\PY{p}{\PYZob{}}
      \PY{c+c1}{// code if boolean is true}
      \PY{p}{\PYZcb{}}\PY{k}{else}\PY{p}{\PYZob{}}
      \PY{c+c1}{// code if vboolean is false}
      \PY{p}{\PYZcb{}}
    \end{Verbatim}
  \item The `while' statement: usage:
    \begin{Verbatim}[commandchars=\\\{\}]
      \PY{k}{while}\PY{p}{(}\PY{n}{boolean}\PY{p}{)}\PY{p}{\PYZob{}}
      \PY{c+c1}{// code to execute WHILE boolean is true}
      \PY{p}{\PYZcb{}}
    \end{Verbatim}
  \end{enumerate}
\end{enumerate}

\newpage

\noindent \textbf{Structure of a Robot Program:}

\begin{enumerate}
\item Create a local representaion of the robot on the machine:
  \begin{Verbatim}[commandchars=\\\{\}]
    \PY{n}{NXT}  \PY{n}{nxt}\PY{p}{;}
    \PY{n}{NXT} \PY{o}{*}\PY{n}{robot} \PY{o}{=} \PY{o}{\PYZam{}}\PY{n}{nxt}\PY{p}{;}
  \end{Verbatim}
\item Initialize the local representation and connect to the robot:
  \begin{Verbatim}[commandchars=\\\{\}]
    \PY{k}{if} \PY{p}{(}\PY{n}{NXT\PYZus{}init}\PY{p}{(}\PY{n}{robot}\PY{p}{)} \PY{o}{|}\PY{o}{|} \PY{n}{NXT\PYZus{}connect}\PY{p}{(}\PY{n}{robot}\PY{p}{,}\PY{l+s}{\PYZdq{}}\PY{l+s}{00:16:53:1A:14:6A}\PY{l+s}{\PYZdq{}}\PY{p}{)}\PY{p}{)}\PY{p}{\PYZob{}}
    \PY{n}{    perror}\PY{p}{(}\PY{l+s}{\PYZdq{}}\PY{l+s}{main: nxt\PYZus{}init}\PY{l+s}{\PYZdq{}}\PY{p}{)}\PY{p}{;}
    \PY{k}{    return} \PY{n}{EXIT\PYZus{}FAILURE}\PY{p}{;}
    \PY{p}{\PYZcb{}}
  \end{Verbatim}
  This code connects to the robot, and exits if something fails while
  connecting.
\item Setup any input devices on the robot:
  \begin{Verbatim}[commandchars=\\\{\}]
    \PY{n}{NXT\PYZus{}initbutton}\PY{p}{(}\PY{n}{robot}\PY{p}{,}\PY{n}{SENSOR\PYZus{}2}\PY{p}{)}\PY{p}{;}
  \end{Verbatim}
  This code says that on `robot' there is a button sensor plugged into port 2.
\item Run the commands.
  \begin{Verbatim}[commandchars=\\\{\}]
    \PY{n}{NXT\PYZus{}driveforward}\PY{p}{(}\PY{n}{robot}\PY{p}{,}\PY{n}{FOREVER}\PY{p}{,}\PY{l+m+mi}{100}\PY{p}{,}\PY{n}{MOTOR\PYZus{}A}\PY{p}{,}\PY{n}{MOTOR\PYZus{}B}\PY{p}{)}\PY{p}{;}
  \end{Verbatim}
  This code says that `robot' should drive forward at 100\% power forever (until
  told to do something else), where the left motor is MOTOR\_A, and the right
  motor is MOTOR\_B.
  \begin{Verbatim}[commandchars=\\\{\}]
    \PY{k}{while}\PY{p}{(}\PY{o}{!}\PY{n}{NXT\PYZus{}ispressed}\PY{p}{(}\PY{n}{robot}\PY{p}{,}\PY{n}{SENSOR\PYZus{}2}\PY{p}{)}\PY{p}{)}\PY{p}{;}
  \end{Verbatim}
  This code says that while the button plugged into port 2 is not being pressed,
  do nothing. This is like waiting for the button to be pressed.
  \begin{Verbatim}[commandchars=\\\{\}]
    \PY{n}{NXT\PYZus{}drivebackward}\PY{p}{(}\PY{n}{robot}\PY{p}{,}\PY{l+m+mi}{3}\PY{p}{,}\PY{l+m+mi}{80}\PY{p}{,}\PY{n}{MOTOR\PYZus{}A}\PY{p}{,}\PY{n}{MOTOR\PYZus{}B}\PY{p}{)}\PY{p}{;}
  \end{Verbatim}
  This code says that after the button has been pressed, `robot' should drive
  backwards at 80\% power for 3 seconds with the same left and right motors as
  before.
  \begin{Verbatim}[commandchars=\\\{\}]
    \PY{n}{NXT\PYZus{}turnleft}\PY{p}{(}\PY{n}{robot}\PY{p}{,}\PY{l+m+mi}{3}\PY{p}{,}\PY{l+m+mi}{100}\PY{p}{,}\PY{n}{MOTOR\PYZus{}A}\PY{p}{,}\PY{n}{MOTOR\PYZus{}B}\PY{p}{)}\PY{p}{;}
  \end{Verbatim}
  This code says to turn `robot' left for 3 seconds, again, with the same
  motors.
\item Close the connection and destroy the local representation of the robot.
  \begin{Verbatim}[commandchars=\\\{\}]
    \PY{n}{NXT\PYZus{}destroy}\PY{p}{(}\PY{n}{robot}\PY{p}{)}\PY{p}{;}
  \end{Verbatim}
  This code says to stop all motors on `robot' and close the connection
  that was opened earlier.
\item Exit the program.
  \begin{Verbatim}[commandchars=\\\{\}]
    \PY{k}{return} \PY{n}{EXIT\PYZus{}SUCCESS}\PY{p}{;}
  \end{Verbatim}
  This says to exit the program with an exit code denoting that everything
  worked as intended.
\end{enumerate}


\newpage

\noindent \textbf{Function Appendix:}
\begin{Verbatim}[commandchars=\\\{\}]
  \PY{c+c1}{// Initializes a previously allocated NXT structure.}
  \PY{c+c1}{// return: 0 on success, non\PYZhy{}zero on failure.}
  \PY{k+kt}{int} \PY{n+nf}{NXT\PYZus{}init}\PY{p}{(}\PY{n}{NXT}\PY{o}{*}\PY{p}{)}\PY{p}{;}

  \PY{c+c1}{// Safely releases all resources held by this NXT structure.}
  \PY{c+c1}{// return: 0 on success, non\PYZhy{}zero on failure.}
  \PY{k+kt}{int} \PY{n+nf}{NXT\PYZus{}destroy}\PY{p}{(}\PY{n}{NXT}\PY{o}{*}\PY{p}{)}\PY{p}{;}

  \PY{c+c1}{// Connects an NXT to a given mac address.}
  \PY{c+c1}{// return: 0 on success, non\PYZhy{}zero on failure.}
  \PY{k+kt}{int} \PY{n+nf}{NXT\PYZus{}connect}\PY{p}{(}\PY{n}{NXT}\PY{o}{*}\PY{p}{,}\PY{k}{const} \PY{k+kt}{char}\PY{o}{*}\PY{p}{)}\PY{p}{;}

  \PY{c+c1}{// Closes the connection of the given NXT.}
  \PY{c+c1}{// return: 0 on success, non\PYZhy{}zero on failure.}
  \PY{k+kt}{int} \PY{n+nf}{NXT\PYZus{}disconnect}\PY{p}{(}\PY{n}{NXT}\PY{o}{*}\PY{p}{)}\PY{p}{;}

  \PY{c+c1}{// Returns the battery level in mV of the NXT.}
  \PY{k+kt}{unsigned} \PY{k+kt}{short} \PY{n+nf}{NXT\PYZus{}battery\PYZus{}level}\PY{p}{(}\PY{n}{NXT}\PY{o}{*}\PY{p}{)}\PY{p}{;}

  \PY{c+c1}{// Tells the nxt that there is a button plugged in on this port.}
  \PY{c+c1}{// return: 0 on success, non\PYZhy{}zero on failure.}
  \PY{k+kt}{int} \PY{n+nf}{NXT\PYZus{}initbutton}\PY{p}{(}\PY{n}{NXT}\PY{o}{*}\PY{p}{,}\PY{n}{sensor\PYZus{}port}\PY{p}{)}\PY{p}{;}

  \PY{c+c1}{// Tells the nxt that there is a light sensor plugged in on this port.}
  \PY{c+c1}{// return: 0 on success, non\PYZhy{}zero on failure.}
  \PY{k+kt}{int} \PY{n+nf}{NXT\PYZus{}initlight}\PY{p}{(}\PY{n}{NXT}\PY{o}{*}\PY{p}{,}\PY{n}{sensor\PYZus{}port}\PY{p}{)}\PY{p}{;}

  \PY{c+c1}{// Queries the button plugged into port p.}
  \PY{c+c1}{// return: true if the button is being pressed, false otherwise.}
  \PY{k+kt}{bool} \PY{n+nf}{NXT\PYZus{}ispressed}\PY{p}{(}\PY{n}{NXT}\PY{o}{*}\PY{p}{,}\PY{n}{sensor\PYZus{}port}\PY{p}{)}\PY{p}{;}

  \PY{c+c1}{// Queries the value of the light sensor plugged into port p.}
  \PY{c+c1}{// return: \PYZhy{}1 on failure, the light value otherwise.}
  \PY{k+kt}{int} \PY{n+nf}{NXT\PYZus{}readlight}\PY{p}{(}\PY{n}{NXT}\PY{o}{*}\PY{p}{,}\PY{n}{sensor\PYZus{}port}\PY{p}{)}\PY{p}{;}

  \PY{c+c1}{// Drive fowards for s seconds at p power}
  \PY{c+c1}{// If s is negative, drive  foward forever.}
  \PY{k+kt}{void} \PY{n+nf}{NXT\PYZus{}driveforward}\PY{p}{(}\PY{n}{NXT}\PY{o}{*}\PY{p}{,}\PY{k+kt}{time\PYZus{}t}\PY{p}{,}\PY{n}{power}\PY{p}{,}\PY{n}{motor\PYZus{}port}\PY{p}{,}\PY{n}{motor\PYZus{}port}\PY{p}{)}\PY{p}{;}

  \PY{c+c1}{// Drive backwards for s seconds at p power.}
  \PY{c+c1}{// If s is negative, drive  foward forever.}
  \PY{k+kt}{void} \PY{n+nf}{NXT\PYZus{}drivebackward}\PY{p}{(}\PY{n}{NXT}\PY{o}{*}\PY{p}{,}\PY{k+kt}{time\PYZus{}t}\PY{p}{,}\PY{n}{power}\PY{p}{,}\PY{n}{motor\PYZus{}port}\PY{p}{,}\PY{n}{motor\PYZus{}port}\PY{p}{)}\PY{p}{;}

  \PY{c+c1}{// Turns the robot left for s seconds (forever if s \PYZlt{} 0).}
  \PY{k+kt}{void} \PY{n+nf}{NXT\PYZus{}turnleft}\PY{p}{(}\PY{n}{NXT}\PY{o}{*}\PY{p}{,}\PY{k+kt}{time\PYZus{}t}\PY{p}{,}\PY{n}{power}\PY{p}{,}\PY{n}{motor\PYZus{}port}\PY{p}{,}\PY{n}{motor\PYZus{}port}\PY{p}{)}\PY{p}{;}

  \PY{c+c1}{// Turns the robot right for s seconds (forever if s \PYZlt{} 0).}
  \PY{k+kt}{void} \PY{n+nf}{NXT\PYZus{}turnright}\PY{p}{(}\PY{n}{NXT}\PY{o}{*}\PY{p}{,}\PY{k+kt}{time\PYZus{}t}\PY{p}{,}\PY{n}{power}\PY{p}{,}\PY{n}{motor\PYZus{}port}\PY{p}{,}\PY{n}{motor\PYZus{}port}\PY{p}{)}\PY{p}{;}

\end{Verbatim}

\newpage

\begin{Verbatim}[commandchars=\\\{\}]

  \PY{c+c1}{// Set the velocity of a motor.}
  \PY{c+c1}{// return: 0 on success, non\PYZhy{}zero on failure.}
  \PY{k+kt}{int} \PY{n+nf}{NXT\PYZus{}setmotor}\PY{p}{(}\PY{n}{NXT}\PY{o}{*}\PY{p}{,}\PY{n}{motor\PYZus{}port}\PY{p}{,}\PY{k+kt}{char}\PY{p}{)}\PY{p}{;}

  \PY{c+c1}{// Stop a motor (set velocity to 0).}
  \PY{c+c1}{// return: 0 on success, non\PYZhy{}zero on failure.}
  \PY{k+kt}{int} \PY{n+nf}{NXT\PYZus{}stopmotor}\PY{p}{(}\PY{n}{NXT}\PY{o}{*}\PY{p}{,}\PY{n}{motor\PYZus{}port}\PY{p}{)}\PY{p}{;}

  \PY{c+c1}{// Stops all motors (sets velocity to 0).}
  \PY{c+c1}{// return: 0 on success, non\PYZhy{}zero on failure.}
  \PY{k+kt}{int} \PY{n+nf}{NXT\PYZus{}stopallmotors}\PY{p}{(}\PY{n}{NXT}\PY{o}{*}\PY{p}{)}\PY{p}{;}

  \PY{c+c1}{// LOW LEVEL FUNCTIONS:}

  \PY{c+c1}{// Initializes all motors by turning them on.}
  \PY{c+c1}{// return: 0 on success, non\PYZhy{}zero on failure.}
  \PY{k+kt}{int} \PY{n+nf}{NXT\PYZus{}initmotors}\PY{p}{(}\PY{n}{NXT}\PY{o}{*}\PY{p}{)}\PY{p}{;}

  \PY{c+c1}{// Sends a STAY\PYZus{}ALIVE message to the NXT. Normal commands do not tell the NXT}
  \PY{c+c1}{// Not to turn off, and thus you should be sure to send one of these every so}
  \PY{c+c1}{// often.}
  \PY{c+c1}{// return: 0 on success, non\PYZhy{}zero on failure.}
  \PY{k+kt}{int} \PY{n+nf}{NXT\PYZus{}stay\PYZus{}alive}\PY{p}{(}\PY{n}{NXT}\PY{o}{*}\PY{p}{)}\PY{p}{;}

  \PY{c+c1}{// Sets a motorstate on the NXT. See motorstate for information on}
  \PY{c+c1}{// constructiong a proper state to send.}
  \PY{c+c1}{// If ans, status will be set to the response of the NXT}
  \PY{c+c1}{// Otherwise, status can be NULL.}
  \PY{c+c1}{// return: 0 on success, non\PYZhy{}zero on failure.}
  \PY{k+kt}{int} \PY{n+nf}{NXT\PYZus{}set\PYZus{}motorstate}\PY{p}{(}\PY{n}{NXT}\PY{o}{*}\PY{p}{,}
  \PY{n}{motorstate}\PY{o}{*}\PY{p}{,}
  \PY{k+kt}{bool}\PY{p}{,}
  \PY{k+kt}{unsigned} \PY{k+kt}{char}\PY{o}{*}\PY{p}{)}\PY{p}{;}

  \PY{c+c1}{// Resets the position of the motor on port (MOTOR\PYZus{}A,MOTOR\PYZus{}B,MOTOR\PYZus{}C,or}
  \PY{c+c1}{// MOTOR\PYZus{}ALL).}
  \PY{c+c1}{// If relative, resets it to the last position before motion, else resets it\PYZsq{}s}
  \PY{c+c1}{// absolute position.}
  \PY{c+c1}{// If ans, status will be set to the response of the NXT}
  \PY{c+c1}{// Otherwise, status can be NULL.}
  \PY{c+c1}{// return: 0 on success, non\PYZhy{}zero on failure.}
  \PY{k+kt}{int} \PY{n+nf}{NXT\PYZus{}reset\PYZus{}motor\PYZus{}position}\PY{p}{(}\PY{n}{NXT}\PY{o}{*}\PY{p}{,}
  \PY{n}{motor\PYZus{}port}\PY{p}{,}
  \PY{k+kt}{bool} \PY{n}{relative}\PY{p}{,}
  \PY{k+kt}{bool} \PY{n}{ans}\PY{p}{,}
  \PY{k+kt}{unsigned} \PY{k+kt}{char}\PY{o}{*}\PY{p}{)}\PY{p}{;}

\end{Verbatim}

\newpage

\begin{Verbatim}[commandchars=\\\{\}]

  \PY{c+c1}{// Gets the motorstate of a given port (MOTOR\PYZus{}A,MOTOR\PYZus{}B,MOTOR\PYZus{}C).}
  \PY{c+c1}{// return: 0 on success, non\PYZhy{}zero on failure.}
  \PY{k+kt}{int} \PY{n+nf}{NXT\PYZus{}get\PYZus{}motorstate}\PY{p}{(}\PY{n}{NXT}\PY{o}{*}\PY{p}{,}
  \PY{n}{motor\PYZus{}port}\PY{p}{,}
  \PY{n}{motorstate}\PY{o}{*}\PY{p}{)}\PY{p}{;}

  \PY{c+c1}{// Sets the input mode of a given port to the type and mode specified.}
  \PY{c+c1}{// If ans, status will be set to the response of the NXT}
  \PY{c+c1}{// Otherwise, status can be NULL.}
  \PY{c+c1}{// return: 0 on success, non\PYZhy{}zero on failure.}
  \PY{k+kt}{int} \PY{n+nf}{NXT\PYZus{}set\PYZus{}input\PYZus{}mode}\PY{p}{(}\PY{n}{NXT}\PY{o}{*}\PY{p}{,}
  \PY{n}{sensor\PYZus{}port}\PY{p}{,}
  \PY{n}{sensor\PYZus{}type}\PY{p}{,}
  \PY{n}{sensor\PYZus{}mode}\PY{p}{,}
  \PY{k+kt}{bool}\PY{p}{,}
  \PY{k+kt}{unsigned} \PY{k+kt}{char}\PY{o}{*}\PY{p}{)}\PY{p}{;}

  \PY{c+c1}{// Returns a pointer to the sensorstate\PYZus{}t of a given sensor port (SENSOR\PYZus{}1,}
  \PY{c+c1}{// SENSOR\PYZus{}2,SENSOR\PYZus{}3,SENSOR\PYZus{}4).}
  \PY{c+c1}{// return: 0 on success, non\PYZhy{}zero on failure.}
  \PY{k+kt}{int} \PY{n+nf}{NXT\PYZus{}get\PYZus{}sensorstate}\PY{p}{(}\PY{n}{NXT}\PY{o}{*}\PY{p}{,}\PY{n}{sensor\PYZus{}port}\PY{p}{,}\PY{n}{sensorstate}\PY{o}{*}\PY{p}{)}\PY{p}{;}
\end{Verbatim}


\end{document}
